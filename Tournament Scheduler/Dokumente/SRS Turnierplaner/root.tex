% Dokumentenkopf 
% Diese Vorlage basiert auf "scrreprt" aus dem koma-script.

\documentclass[
   11pt, % Schriftgröße
   DIV10,
   ngerman, % für Umlaute, Silbentrennung etc.
   a4paper, % Papierformat
   oneside, % einseitiges Dokument
   titlepage, % es wird eine Titelseite verwendet
   parskip=half, % Abstand zwischen Absätzen (halbe Zeile)
   headings=normal, % Grˆfle der Überschriften verkleinern
   listof=totoc, % Verzeichnisse im Inhaltsverzeichnis aufführen
   bibliography=totoc, % Literaturverzeichnis im Inhaltsverzeichnis aufführen
   index=totoc, % Index im Inhaltsverzeichnis aufführen
   captions=tableheading, % Beschriftung von Tabellen unterhalb ausgeben
   final, % Status des Dokuments (final/draft)
   numbers=noenddot
]{scrreprt}

%Abstände ändern
\renewcommand*{\chapterheadstartvskip}{\vspace{-1\baselineskip}} %Abstand vor Section
\setlength{\headsep}{30pt} %Abstand Kopfzeile Textkörper

% Meta-Informationen 
%   Informationen über das Dokument, wie z.B. Titel, Autor, Matrikelnr. etc
\input{Meta} %Meta.tex

% benötigte Packages
\usepackage{scrpage2}
\usepackage[utf8]{inputenc}
\usepackage{tabularx}

\usepackage{tikz}
\usetikzlibrary{positioning}
\usepackage{pgf} %Packages.tex
 
% Erstellung eines Index
\makeindex

% Kopf- und Fuflzeilen, Seitenränder etc. 
\input{Seitenstil} %Seitenstil.tex

% eigene Definitionen für Silbentrennung
\include{Silbentrennung} %Silbentrennung.tex

% eigene LaTeX-Befehle
\include{Befehle} %Befehle.tex

% Das eigentliche Dokument
\begin{document}

\setcounter{secnumdepth}{3}
\setcounter{tocdepth}{3}

% Deckblatt und Abstract ohne Seitenzahl
\ofoot{}
\title{Software Requirements Specification}
\author{Erik Koltermann, Julia Geißler}
\date{\today}
\maketitle %Deckblatt.tex
\include{Inhalt/Abstract}
\ofoot{\pagemark}

% Seitennummerierung 
\pagenumbering{Roman}
\tableofcontents % Inhaltsverzeichnis

% Abkürzungsverzeichnis
\include{Inhalt/Akronyme} %Akronyme.tex
\listoffigures % Abbildungsverzeichnis
\listoftables % Tabellenverzeichnis

% arabische Seitenzahlen im Hauptteil 
\clearpage
\pagenumbering{arabic}

% die Inhaltskapitel werden in "Inhalt.tex" inkludiert
\chapter{Einleitung}
\section{Zweck} %Inhalt.tex

% Literaturverzeichnis 
\bibliography{Bibliographie} % Aufruf: bibtex Bibliographie.bib
\bibliographystyle{natdin} % DIN-Stil des Literaturverzeichnisses

% Selbständigkeitserklärung
\include{Erklaerung} %Erklaerung.tex

% Anhang 
%\begin{appendix}
%   \clearpage
%   \pagenumbering{roman}
%   \chapter{Anhang} %Anhang.tex
%   \label{sec:Anhang}
%   \setdefaultleftmargin{1em}{}{}{}{}{}
%   \input{Anhang}
%\end{appendix}

% Ende des Dokuments
\end{document}
------------------------------------------