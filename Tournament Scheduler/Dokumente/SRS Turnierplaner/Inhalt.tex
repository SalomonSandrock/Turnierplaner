\chapter{Einleitung}
\section{Zweck}
Ziel des Projektes ist die Erstellung einer Applikation für iOS zur Planung eines Turniers, möglichst für mehrere Sportarten. Dies beinhaltet grob verschiedene wählbare Turniermodi und die Verwaltung und Pflege einer Datenbank für teilnehmende Teams bzw. Spieler.

\section{Umfang}
Am Ende des Projekts soll eine lauffähige Applikation stehen, mit dessen Hilfe man ein Turnier verschiedener Sportarten planen und durchführen kann. Dabei können Turniere, Mannschaften und einzelne Spieler hinzugefügt werden.
Während eines Turniers können die Endstände eingetragen werden und es wird angezeigt, welche Spiele als nächstes statt finden. 

Sowohl eine Einzelansicht und Auswertung pro Mannschaft als auch pro Spieler sollte verfügbar sein, um nicht nur die Teamleistung, sondern auch die individuellen Leistungen der Spieler (beispielsweise das Erzielen der meisten Punkte/Tore) zu bewerten. Mit der Team-Ansicht kann auf gezielte Fragen der Teilnehmer reagiert werden: wann ihr nächstes Spiel stattfindet, wo ihre bisherigen Leistungen im gesamten Teilnehmerfeld einzuordnen sind und was sie dementsprechend noch erreichen können etc. Dies ist insbesondere bei sehr großen Turnieren von Vorteil, um nicht den Überblick zu verlieren. \\
Einige dieser Funktionalitäten sind nur sekundärer Natur, im Vordergrund steht natürlich anfangs die brauchbare Verwertung der Spielstände und die Umsetzung verschiedenster Spielmodi, sodass die App auch ohne diese Zusatzfunktionen, die im Laufe der Zeit noch eingepflegt werden können, lauffähig und benutzbar ist. 
Die Einteilung in die unterschiedlichen Prioritätsstufen erfolgt jedoch zu einem späteren Zeitpunkt.
\section{Begriffe und Abkürzungen}

	\begin{tabular}{| l | l |}
		\hline
		Abkürzung & Bedeutung\\
		\hline
		App & Applikation\\
		User & Benutzer der Applikation \\
		\hline
	\end{tabular}

\section{Aufbau des Dokuments}

\chapter{Allgemeine Beschreibung des Projekts}
\section{Produktperspektive (zu anderen Produkten)}
In der Zukunft kann die App einerseits in ihrem Umfang aber auch ihrer Laufwerksumgebung erweitert werden.
Zur Zeit sieht das Projekt nur eine Realisierung für iOS-Geräte vor. Hierbei wird die App vorerst auch nur für größere Bildschirme optimiert, wie zum Beispiel dem iPad. Allerdings ist es durchaus möglich später auch eine Realisierung für andere Betriebssysteme vorzunehmen.
Mit der Zeit werden auch weitere Sportarten integriert. Die App wird zunächst allgemein entwickelt, sodass viele Sportarten abgedeckt werden können. Jedoch werden Spezialisierungen nach und nach eingefügt.
 
\section{Übersicht der Produktfunktionen}
\section{Benutzermerkmale}
Zur Benutzung der App benötigt der User ein iOS-fähiges Gerät und eine registrierte Apple-ID.

\section{Einschränkungen für den Entwickler}
\section{Annahmen und Abhängigkeiten}
\section{Aufteilung und Priorisierung der Anforderungen}

\chapter{Spezifische Anforderungen}
Die Anforderungen der App werden hauptsächlich im Kapitel: "Funktionale Anforderungen" aufgeführt. Jeder Anforderung wird eine ID zugeordnet, welche sie auch in Zukunft kenntlich machen wird.
Desweiteren werden die Anforderungen durch ein Zahlensystem von 1 bis 5 priorisiert. Dabei steht die 1 für die höchste Priorität und die 5 für die geringste Priorität. Diese Prioritäten helfen später bei der Umsetzung. Sind die Anforderungen mit der Priorität 1 und 2 abgearbeitet, so ist die App grundsätzlich lauffähig. Alle weiteren Anforderungen können durch Updates hinzugefügt  werden. Durch dieses Vorgehen wird sichergestellt, dass die App schnellstmöglich in Anwendung gehen kann.

\section{Funktionale Anforderungen}
	\begin{itemize}
		\item[0100] Hinzufügen eines neuen Turniers
		\item[0200] Turnier auswählen
		\item[0300] Name eines Turniers ändern
		\item[0400] Turnier löschen
		\item[0500] Mannschaft einem Turnier hinzufügen
		\item[0600] Mannschaft aus einem Turnier entfernen
		\item[0700] Spieler zu einer Mannschaft hinzufügen
		\item[0800] Spieler aus einer Mannschaft entfernen
		\item[0900] Eigenschaften eines Turniers wählen/ändern
		\begin{itemize}
			\item[0910] Sportart einstellen
			\item[0920] Spielmodus einstellen
		\end{itemize}
		\item[1000] Turnier starten
		\item[1100] Anzeige der Spiele
		\item[1200] Spielergebnisse eintragen
	\end{itemize}
	
	\begin{tabularx}{\textwidth}{|X|X|X|}
		\hline
		ID & Priorität & Abhängigkeit \\
		0100 & 1 & \\
		\hline
		\multicolumn{3}{|l|}{Name: Hinzufügen eines Turniers} \\
		\hline		
		\multicolumn{3}{|l|}{Beschreibung:}\\
		\multicolumn{3}{|l|}{Im Startfenster kann ein Turnier durch Betätigung des Buttons "+" ein Turnier hinzugefügt werden. Ein Popp up Fenster mit Eingabefeld erscheint.}\\
		\hline		 
	\end{tabularx}
	
	\begin{tabularx}{\textwidth}{|X|X|X|}
		\hline
		ID & Priorität & Abhängigkeit \\
		0200 & 1 & 0100\\
		\hline
		\multicolumn{3}{|l|}{Name: Turnier auswählen} \\
		\hline
		\multicolumn{3}{|l|}{Beschreibung}\\
		\multicolumn{3}{|l|}{}\\
		\hline		 
	\end{tabularx}
	
	\begin{tabularx}{\textwidth}{|X|X|X|}
		\hline
		ID & Priorität & Abhängigkeit \\
		0300 & 2 & 0200\\
		\hline
		\multicolumn{3}{|l|}{Name: eines Turniers ändern} \\
		\hline
		\multicolumn{3}{|l|}{Beschreibung}\\
		\multicolumn{3}{|l|}{}\\
		\hline		 
	\end{tabularx}
	
	\begin{tabularx}{\textwidth}{|X|X|X|}
		\hline
		ID & Priorität & Abhängigkeit \\
		0400 & 3 & 0200\\
		\hline
		\multicolumn{3}{|l|}{Name: Turnier löschen} \\
		\hline
		\multicolumn{3}{|l|}{Beschreibung}\\
		\multicolumn{3}{|l|}{}\\
		\hline		 
	\end{tabularx}
	
	\begin{tabularx}{\textwidth}{|X|X|X|}
		\hline
		ID & Priorität & Abhängigkeit \\
		0500 & 1 & 0200\\
		\hline
		\multicolumn{3}{|l|}{Name: Mannschaft einem Turnier hinzufügen} \\
		\hline
		\multicolumn{3}{|l|}{Beschreibung}\\
		\multicolumn{3}{|l|}{}\\
		\hline		 
	\end{tabularx}
	
	\begin{tabularx}{\textwidth}{|X|X|X|}
		\hline
		ID & Priorität & Abhängigkeit \\
		0600 & 2 & 0500\\
		\hline
		\multicolumn{3}{|l|}{Name: Mannschaft aus einem Turnier entfernen} \\
		\hline
		\multicolumn{3}{|l|}{Beschreibung}\\
		\multicolumn{3}{|l|}{}\\
		\hline		 
	\end{tabularx}
	
	\begin{tabularx}{\textwidth}{|X|X|X|}
		\hline
		ID & Priorität & Abhängigkeit \\
		0700 & 3 & 0500\\
		\hline
		\multicolumn{3}{|l|}{Name: Spieler einer Mannschaft hinzufügen} \\
		\hline
		\multicolumn{3}{|l|}{Beschreibung}\\
		\multicolumn{3}{|l|}{}\\
		\hline		 
	\end{tabularx}
	
	\begin{tabularx}{\textwidth}{|X|X|X|}
		\hline
		ID & Priorität & Abhängigkeit \\
		0800 & 4 & 0700\\
		\hline
		\multicolumn{3}{|l|}{Name: Spieler aus einer Mannschaft entfernen} \\
		\hline
		\multicolumn{3}{|l|}{Beschreibung}\\
		\multicolumn{3}{|l|}{}\\
		\hline		 
	\end{tabularx}
	
	\begin{tabularx}{\textwidth}{|X|X|X|}
		\hline
		ID & Priorität & Abhängigkeit \\
		0900 & 1 & 0500\\
		\hline
		\multicolumn{3}{|l|}{Name: Eigenschaften eines Turniers wählen/ändern} \\
		\hline
		\multicolumn{3}{|l|}{Beschreibung}\\
		\multicolumn{3}{|l|}{}\\
		\hline		 
	\end{tabularx}
	
	\begin{tabularx}{\textwidth}{|X|X|X|}
		\hline
		ID & Priorität & Abhängigkeit \\
		0910 & 3 & 0900\\
		\hline
		\multicolumn{3}{|l|}{Name: Sportart auswählen} \\
		\hline
		\multicolumn{3}{|l|}{Beschreibung}\\
		\multicolumn{3}{|l|}{}\\
		\hline		 
	\end{tabularx}
	
	\begin{tabularx}{\textwidth}{|X|X|X|}
		\hline
		ID & Priorität & Abhängigkeit \\
		0920 & 2 & 0900\\
		\hline
		\multicolumn{3}{|l|}{Name: Spielmodus einstellen} \\
		\hline
		\multicolumn{3}{|l|}{Beschreibung}\\
		\multicolumn{3}{|l|}{}\\
		\hline		 
	\end{tabularx}
	
	\begin{tabularx}{\textwidth}{|X|X|X|}
		\hline
		ID & Priorität & Abhängigkeit \\
		1000 & 1 & 0900\\
		\hline
		\multicolumn{3}{|l|}{Name: Turnier starten} \\
		\hline
		\multicolumn{3}{|l|}{Beschreibung}\\
		\multicolumn{3}{|l|}{Durch Betätigung eines Buttens "Turnier starten" wird das Turnier gestartet und die Ansetzungen angezeigt. Von nun an ist es nicht mehr möglich den Modus des Turniers, die Sportart und die Teams zu ändern. }\\
		\hline		 
	\end{tabularx}
	
	\begin{tabularx}{\textwidth}{|X|X|X|}
		\hline
		ID & Priorität & Abhängigkeit \\
		1100 & 1 & 1000\\
		\hline
		\multicolumn{3}{|l|}{Name: Anzeige der Spiele} \\
		\hline
		\multicolumn{3}{|l|}{Beschreibung}\\
		\multicolumn{3}{|l|}{}\\
		\hline		 
	\end{tabularx}
	
	\begin{tabularx}{\textwidth}{|X|X|X|}
		\hline
		ID & Priorität & Abhängigkeit \\
		1200 & 1 & 1100\\
		\hline
		\multicolumn{3}{|l|}{Name: Spielergebnisse eintragen} \\
		\hline
		\multicolumn{3}{|l|}{Beschreibung}\\
		\multicolumn{3}{|l|}{}\\
		\hline		 
	\end{tabularx}
	
	\begin{tikzpicture} [node distance=0.75cm and 0.5cm, rounded corners, >=stealth]
		\node[draw, rectangle] (0000) {start};
		\node[draw, rectangle, below =of 0000] (0100) {0100};
		\node[draw, rectangle, below =of 0100] (0200) {0200};
		\node[draw, rectangle, below left=of 0200] (0300) {0300};
		\node[draw, rectangle, below right=of 0200] (0400) {0400};
		\node[draw, rectangle, below =of 0200] (0500) {0500};
		\node[draw, rectangle, below left=of 0500] (0600) {0600};
		\node[draw, rectangle, below right=of 0500] (0700) {0700};
		\node[draw, rectangle, below right=of 0700] (0800) {0800};
		\node[draw, rectangle, below =of 0500] (0900) {0900};
		\node[draw, rectangle, below left=of 0900] (0910) {0910};
		\node[draw, rectangle, below right=of 0900] (0920) {0920};
		\node[draw, rectangle, below =of 0900] (1000) {1000};
		\node[draw, rectangle, below =of 1000] (1100) {1100};
		\node[draw, rectangle, below =of 1100] (1200) {1200};
		
		\draw[->, >=stealth] (0000) -- (0100);
		\draw[->, >=stealth] (0100) -- (0200);
		\draw[->, >=stealth, rounded corners] (0200.south) -- ++(0, -0.5)-| (0300);
		\draw[->, >=stealth, rounded corners] (0200.south) -- ++(0, -0.5)-| (0400);
		\draw[->, >=stealth] (0200) -- (0500);
		\draw[->, >=stealth, rounded corners] (0500.south) -- ++(0, -0.5)-| (0600);
		\draw[->, >=stealth, rounded corners] (0500.south) -- ++(0, -0.5)-| (0700);
		\draw[->, >=stealth, rounded corners] (0700.east) -- ++(0, -0.5)-| (0800);
		\draw[->, >=stealth] (0500) -- (0900);
		\draw[->, >=stealth, rounded corners] (0900.south) -- ++(0, -0.5)-| (0910);
		\draw[->, >=stealth, rounded corners] (0900.south) -- ++(0, -0.5)-| (0920);
		\draw[->, >=stealth] (0900) -- (1000);
		\draw[->, >=stealth] (1000) -- (1100);
		\draw[->, >=stealth] (1100) -- (1200);
	\end{tikzpicture}

\section{Nicht-funktionale Anforderungen}
	\begin{itemize}
		\item[1] Start der App innerhalb von 0.5 Sekunden
		\item[2] Verzeichnistiefe nicht höher als 3
	\end{itemize}
	
\section{Externe Schnittstellen}
\section{Design Constraints}
\section{Anforderungen an Performance}
\section{Qualitätsanforderungen}
\section{Sonstige Anforderungen}
