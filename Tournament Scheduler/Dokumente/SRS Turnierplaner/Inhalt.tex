\chapter{Einleitung}
\section{Zweck}
Ziel des Projektes ist die Erstellung einer Applikation für iOS zur Planung eines Turniers, möglichst für mehrere Sportarten. Dies beinhaltet grob verschiedene wählbare Turniermodi und die Verwaltung und Pflege einer Datenbank für teilnehmende Teams bzw. Spieler.

\section{Umfang}
Am Ende des Projekts soll eine lauffähige Applikation stehen, mit dessen Hilfe man ein Turnier verschiedener Sportarten planen und durchführen kann. Dabei können Turniere, Mannschaften und einzelne Spieler hinzugefügt werden.
Während eines Turniers können die Endstände eingetragen werden und es wird angezeigt, welche Spiele als nächstes statt finden.

\section{Begriffe und Abkürzungen}
	\begin{table}{l l}
		Abkürzung & Bedeutung\\
		App & Applikation\\
		User & Benutzer der Applikation\\
	\end{table}

\section{Aufbau des Dokuments}

\chapter{Allgemeine Beschreibung des Projekts}
\section{Produktperspektive (zu anderen Produkten)}
In der Zukunft kann die App einerseits in ihrem Umfang aber auch ihrer Laufwerksumgebung erweitert werden.
Zur Zeit sieht das Projekt nur eine Realisierung für iOS-Geräte vor. Hierbei wird die App vorerst auch nur für größere Bildschirme optimiert, wie zum Beispiel dem iPad. Allerdings ist es durchaus möglich später auch eine Realisierung für andere Betriebssysteme vorzunehmen.
Mit der Zeit werden auch weitere Sportarten integriert. Die App wird zunächst allgemein entwickelt, sodass viele Sportarten abgedeckt werden können. Jedoch werden Spezialisierungen nach und nach eingefügt.
 
\section{Übersicht der Produktfunktionen}
\section{Benutzermerkmale}
Zur Benutzung der App benötigt der User ein iOS-fähiges Gerät und eine registrierte Apple-ID.

\section{Einschränkungen für den Entwickler}
\section{Annahmen und Abhängigkeiten}
\section{Aufteilung und Priorisierung der Anforderungen}

\chapter{Spezifische Anforderungen}
\section{Funktionale Anforderungen}
	\begin{tabularx}{\textwidth}{|l|l|l|}
		ID & Priorität & Abhängigkeit \\
		   &           &              \\
		\hline
		\multicolumn{3}{l}{Beschreibung}\\
		\multicolumn{3}{l}{}\\
		\hline		 
	\end{tabular}

\section{Nicht-funktionale Anforderungen}
	\begin{itemize}
		\item[1] Start der App innerhalb von 0.5 Sekunden
		\item[2] Verzeichnistiefe nicht höher als 3
	\end{itemize}
	
\section{Externe Schnittstellen}
\section{Design Constraints}
\section{Anforderungen an Performance}
\section{Qualitätsanforderungen}
\section{Sonstige Anforderungen}
